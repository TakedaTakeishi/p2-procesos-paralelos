% ==========================================
% ANEXOS — ARCHIVO SEMI-ESTÁTICO
% ==========================================
% Agrega tus apéndices aquí. A diferencia de los otros
% archivos en common/, este SÍ se edita por proyecto.
%
% TIP: Los anexos se numeran con letras (A, B, C...).
%      Usa \chapter{} para cada anexo y \section{} para subdivisiones.
%
% TIP: Cada anexo que agregues aparecerá automáticamente
%      en el índice general.
% ==========================================

\appendix
\appendixpage

% ==========================================
% EJEMPLO DE ANEXO — Elimina o reemplaza según necesites
% ==========================================
\chapter{Anexo de ejemplo}
\label{anexo:ejemplo}

Este es un anexo de ejemplo. Puedes agregar aquí información
complementaria como diagramas grandes, código extenso,
tablas de datos, manuales, etc.

% TIP: Ejemplo de figura grande en un anexo:
% \begin{figure}[H]
%     \centering
%     \includegraphics[width=\textwidth,height=0.85\textheight,keepaspectratio]{nombre_imagen.png}
%     \caption{Descripción de la imagen}
%     \label{fig:anexo_imagen}
% \end{figure}

% TIP: Ejemplo de tabla larga en un anexo:
% \begin{longtable}{@{}p{3cm}p{3cm}p{6cm}@{}}
% \toprule
% \textbf{Columna 1} & \textbf{Columna 2} & \textbf{Columna 3} \\ \midrule
% \endhead
% Dato 1 & Dato 2 & Dato 3 \\
% Dato 4 & Dato 5 & Dato 6 \\ \bottomrule
% \caption{Título de la tabla}
% \label{tab:anexo_tabla}
% \end{longtable}

% TIP: Para agregar más anexos, simplemente agrega más \chapter{}:
% \chapter{Segundo Anexo}
% \label{anexo:segundo}
% Contenido del segundo anexo...